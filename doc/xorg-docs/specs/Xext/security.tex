% $Xorg: security.tex,v 1.3 2000/08/17 19:42:36 cpqbld Exp $
\documentstyle{article}
\pagestyle{myheadings}
\markboth{Security Extension Specification}{Security Extension Specification}
\setlength{\parindent}{0 pt}
\setlength{\parskip}{6pt}
\setlength{\topsep}{0 pt}

% Request names are literal symbols; therefore, use the same font for both.
\newcommand{\requestname}[1]{{\tt #1}}
\newcommand{\literal}[1]{\mbox{\tt #1}}

\newcommand{\encodingsection}[1]{{\bf #1}}
\newcommand{\requestsection}[1]{{\bf #1}}

% Font treatment of type names differs between protocol and library sections.
\newcommand{\libtypename}[1]{\mbox{\tt #1}}
\newcommand{\typename}[1]{\mbox{\rm #1}}  % default font
\newcommand{\typeargname}[1]{\mbox{\rm #1}}  % default font
\newcommand{\argname}[1]{\mbox{\it #1}}
\newcommand{\argdecl}[2]{\argname{#1} & : \typename{#2}\\}
\newcommand{\areplyargdecl}[2]{#1 & : \typename{#2}\\}

\newenvironment{arequest}[1]{\requestsection{#1} \\ \begin{tabular}{ll}}{\end{tabular}}
\newcommand{\areply}{$\Rightarrow$\\}

\newcommand{\etabstops}{\hspace*{0cm}\=\hspace*{1cm}\=\hspace*{2.5cm}\=\hspace*{2.5cm}\=\kill}

\newcommand{\eargdecl}[3]{\> #1 \> \typename{#2} \> \> #3 \\}
\newcommand{\econstdecl}[2]{\> \> #1 \> #2 \\}

\newenvironment{keeptogether}{\vbox \bgroup}{\egroup}

\newenvironment{erequest}[3]{\pagebreak[3] \begin{keeptogether} \encodingsection{#1} \begin{tabbing} \etabstops \eargdecl{1}{CARD8}{major-opcode} \eargdecl{1}{#2}{minor-opcode} \eargdecl{2}{#3}{request length}}{\end{tabbing} \end{keeptogether}}

\newenvironment{estruct}[1]{\pagebreak[3] \begin{keeptogether} \encodingsection{#1} \begin{tabbing} \etabstops }{\end{tabbing} \end{keeptogether}}

\newenvironment{eerror}[1]{\begin{keeptogether} \encodingsection{#1} \begin{tabbing} \etabstops }{\end{tabbing} \end{keeptogether}}

\newenvironment{etypedef}[1]{\begin{keeptogether} \typename{#1} \begin{tabbing} \etabstops }{\end{tabbing} \end{keeptogether}}

\newcommand{\cfunctionname}[1]{\mbox{\tt #1}}
\newcommand{\cfunctiondecl}[1]{\mbox{\rm #1}}
\newcommand{\cargdecl}[2]{\penalty -1\typename{#1} \argname{#2}}
\newenvironment{cfunction}[2]{\begin{sloppypar}\begin{keeptogether}\vspace{5mm}\typename{#1}\\ \cfunctiondecl{#2}\ (}{)\end{keeptogether}\end{sloppypar}{\hangafter=2 \hangindent=20pt \raggedright\par}}

% make things easier with all the long names
\spaceskip .3333em plus 5em
\tolerance=2000

\begin{document}

\title{Security Extension Specification\\Version 7.1\\X11 Release 6.8}
\author{David P. Wiggins\\X Consortium, Inc.}
\date{November 15, 1996}
\maketitle
\thispagestyle{empty}

\eject

Copyright \copyright 1996 X Consortium, Inc.

THE SOFTWARE IS PROVIDED "AS IS", WITHOUT WARRANTY OF ANY KIND,
EXPRESS OR IMPLIED, INCLUDING BUT NOT LIMITED TO THE WARRANTIES OF
MERCHANTABILITY, FITNESS FOR A PARTICULAR PURPOSE AND NONINFRINGEMENT.
IN NO EVENT SHALL THE X CONSORTIUM BE LIABLE FOR ANY CLAIM, DAMAGES OR
OTHER LIABILITY, WHETHER IN AN ACTION OF CONTRACT, TORT OR OTHERWISE,
ARISING FROM, OUT OF OR IN CONNECTION WITH THE SOFTWARE OR THE USE OF
OR OTHER DEALINGS IN THE SOFTWARE.

Except as contained in this notice, the name of the X Consortium shall
not be used in advertising or otherwise to promote the sale, use or
other dealings in this Software without prior written authorization
from the X Consortium.
\eject

\section{Introduction}

The Security extension contains new protocol needed to provide
enhanced X server security.  The Security extension should not be
exposed to untrusted clients (defined below).

\section{Requests}

\subsection{SecurityQueryVersion}

This request returns the major and minor version numbers of this
extension.

% start marker
\begin{arequest}{SecurityQueryVersion}
\argdecl{client-major-version}{CARD16}
\argdecl{client-minor-version}{CARD16}
\areply
\areplyargdecl{server-major-version}{CARD16}
\areplyargdecl{server-minor-version}{CARD16}
\end{arequest}
% end marker

The client-major-version and client-minor-version
numbers indicate what version of the protocol the client wants the
server to implement.  The server-major-version and the
server-minor-version numbers returned indicate the protocol
this extension actually supports.  This might not equal the version
sent by the client.  An implementation can (but need not) support more
than one version simultaneously.  The server-major-version
and server-minor-version allow the creation of future
revisions of the Security protocol that may be
necessary.  In general, the major version would increment for
incompatible changes, and the minor version would increment for small,
upward-compatible changes.  Servers that support the protocol defined
in this document will return a server-major-version of one
(1), and a server-minor-version of zero (0).

Clients using the Security extension must issue a
\requestname{SecurityQueryVersion} request before any other Security
request in order to negotiate a compatible protocol version;
otherwise, the client will get undefined behavior (Security may or may
not work).

\subsection{SecurityGenerateAuthorization}

This request causes the server to create and return a new
authorization with specific characteristics.  Clients can subsequently
connect using the new authorization and will inherit some of the
characteristics of the authorization.

% start marker
\begin{arequest}{SecurityGenerateAuthorization}
\argdecl{authorization-protocol-name}{STRING8}
\argdecl{authorization-protocol-data}{STRING8}
\argdecl{value-mask}{BITMASK}
\argdecl{value-list}{LISTofVALUE}
\areply
\areplyargdecl{authorization-id}{AUTHID}
\areplyargdecl{authorization-data-return}{STRING8}
\end{arequest}

Errors: \literal{AuthorizationProtocol, Value, Alloc}
% end marker

authorization-protocol-name is the name of the authorization method
for which the server should generate a new authorization that
subsequent clients can use to connect to the server.  If the
authorization-protocol-name is not one that the server supports, or if
authorization-protocol-data does not make sense for the given
authorization-protocol-name, an AuthorizationProtocol error results.

authorization-protocol-data is authorization-method specific data that
can be used in some way to generate the authorization.

Note: in this version of the extension, the only authorization method
required to be supported is ``MIT-MAGIC-COOKIE-1'' with any amount of
authorization-protocol-data (including none).  The server may use the
authorization-protocol-data as an additional source of randomness used
to generate the authorization.  Other authorization methods can supply
their own interpretation of authorization-protocol-data.

The value-mask and value-list specify attributes of the authorization
that are to be explicitly initialized.  The possible values are:

\begin{tabular}{lll}
\\ \hline
Attribute	& Type		& Default	\\ \hline
timeout		& CARD32	& 60		\\
group		& XID or None	& None		\\
trust-level	& \{SecurityClientTrusted, \\
		& SecurityClientUntrusted\} & SecurityClientUntrusted \\
event-mask	& SecurityAuthorizationRevoked, \\
		& or None	& None		\\ \hline
\\
\end{tabular}

timeout is the timeout period in seconds for this authorization.  A
timeout value of zero means this authorization will never expire.  For
non-zero timeout values, when timeout seconds have elapsed since the
last time that the authorization entered the state of having no
connections authorized by it, and if no new connections used the
authorization during that time, the authorization is automatically
purged.  (Note that when an authorization is created, it enters the
state of having no connections authorized by it.)  Subsequent
connection attempts using that authorization will fail.  This is to
facilitate ``fire and forget'' launching of applications.

group is an application group ID as defined by the Application Group
extension, or None.  Any other values will cause a Value error.  When
a group is destroyed, all authorizations specifying that group are
revoked as described under the SecurityRevokeAuthorization request.
The Application Group extension attaches additional semantics to the
group.

trust-level tells whether clients using the authorization are trusted
or untrusted.  If trust-level is not one of the constants
SecurityClientTrusted or SecurityClientUntrusted, a Value error
results.

event-mask defines which events the client is interested in for
this authorization.  When the authorization expires or is revoked
if event-mask contains SecurityAuthorizationRevoked a
SecurityAuthorizationRevoked event is reported to the client.

The SecurityAuthorizationRevoked event contains the following field:

\begin{tabular}{ll}
\\ \hline
Field	& Type	\\ \hline
authorization-id	& AUTHID	\\ \hline
\\
\end{tabular}

where authorization-id is the identification of the authorization
that was revoked.

If an invalid value-mask is specified, a Value error occurs.

The returned authorization-id is a non-zero value that uniquely
identifies this authorization for use in other requests.  The value
space for type AUTHID is not required to be disjoint from values
spaces of other core X types, e.g. resource ids, atoms, visual ids,
and keysyms.  Thus, a given numeric value might be both a valid
AUTHID and a valid atom, for example.

authorization-data-return is the data that a client should use in some
authorization-method-specific way to make a connection with this
authorization.  For ``MIT-MAGIC-COOKIE-1,'' authorization-data-return
should be sent as the authorization-protocol-data in the connection setup
message.  It is not required that other authorization methods use
authorization-data-return this way.

\subsection{SecurityRevokeAuthorization}

This request deletes an authorization created by
SecurityGenerateAuthorization.

% start marker
\begin{arequest}{SecurityRevokeAuthorization}
\argdecl{authorization-id}{AUTHID}
\end{arequest}

Errors: \literal{Authorization}
% end marker

If authorization-id does not name a valid authorization, an
Authorization error occurs.  Otherwise, this request kills all clients
currently connected using the authorization specified by
authorization-id.  The authorization is deleted from the server's
database, so future attempts by clients to connect with this
authorization will fail.

\section{Changes to Core Requests}

A server supporting this extension modifies the handling of some core
requests in the following ways.

\subsection{Resource ID Usage}

If an untrusted client makes a request that specifies a resource ID
that is not owned by another untrusted client, a protocol error is
sent to the requesting client indicating that the specified resource
does not exist.  The following exceptions apply.  An untrusted client
can:

\begin{enumerate}

\item use the QueryTree, GetGeometry, and TranslateCoordinates requests
without restriction.
\item use colormap IDs that are returned in the default-colormap
field of its connection setup information in any colormap requests.
\item specify a root window as:

\begin{enumerate}

\item the drawable field of CreatePixmap, CreateGC, and QueryBestSize.
\item the parent field of CreateWindow.
\item the window field of CreateColormap, ListProperties, and GetWindowAttributes.
\item the grab-window or confine-to fields of GrabPointer.
\item the grab-window field of UngrabButton.
\item the destination of SendEvent, but only if all of the following
are true.  (These conditions cover all the events that the ICCCM
specifies with a root window destination.)

\begin{enumerate}

\item The propogate field of SendEvent is False.
\item The event-mask field of SendEvent is ColormapChange,
StructureNotify, or the logical OR of SubstructureRedirect with
SubstructureNotify.
\item The event type being sent is UnmapNotify, ConfigureRequest,
or ClientMessage.

\end{enumerate}

\item the window field of ChangeWindowAttributes, but only if
the value-mask contains only event-mask and the corresponding value
is StructureNotify, PropertyChange, or the logical OR of both.

\end{enumerate}
\end{enumerate}

ISSUE: are root window exceptions needed for these?  WarpPointer,
ReparentWindow (parent), CirculateWindow, QueryPointer (emacs does
this), GetMotionEvents.

\subsection{Extension Security}

This extension introduces the notion of secure and insecure
extensions.  A secure extension is believed to be safe to use by
untrusted clients; that is, there are no significant security concerns
known that an untrusted client could use to destroy, modify, or steal
data of trusted clients.  This belief may be founded on a careful
analysis of the extension protocol, its implementation, and measures
taken to ``harden'' the extension to close security weaknesses.  All
extensions not considered secure are called insecure.  The
implementation details of how an extension is identified as
as secure or insecure are beyond the scope of this specification.

ListExtensions will only return names of secure extensions to
untrusted clients.

If an untrusted client uses QueryExtension on an insecure extension
that the server supports, the reply will have the present field set to
False and the major-opcode field set to zero to indicate that the
extension is not supported.

If an untrusted client successfully guesses the major opcode of an
insecure extension, attempts by it to execute requests with that major
opcode will fail with a Request error.

\subsection{Keyboard Security}

The protocol interpretation changes in this section are intended to
prevent untrusted applications from stealing keyboard input that was
meant for trusted clients and to prevent them from interfering with
the use of the keyboard.

The behavior of some keyboard-related requests and events is modified
when the client is untrusted depending on certain server state at the
time of request execution or event generation.  Specifically, if a
hypothetical keyboard event were generated given the current input
focus, pointer position, keyboard grab state, and window event
selections, and if that keyboard event would not be delivered to any
untrusted client, the following changes apply:

\begin{enumerate}

\item The bit vector representing the up/down state of the keys
returned by QueryKeymap and KeymapNotify is all zeroes.
\item GrabKeyboard returns a status of AlreadyGrabbed.
\item SetInputFocus does nothing.  Note that this means
the Globally Active Input and WM\_TAKE\_FOCUS mechanisms specified
in the ICCCM will not work with untrusted clients.
\item Passive grabs established by GrabKey that would otherwise have
activated do not activate.

\end{enumerate}

If an untrusted client attempts to use any of the following requests,
the only effect is that the client receives an Access error:
SetModifierMapping, ChangeKeyboardMapping, ChangeKeyboardControl.

If an InputOnly window owned by an untrusted client has a parent owned
by a trusted client, all attempts to map the window will be ignored.
This includes mapping attempts resulting from MapWindow,
MapSubwindows, ReparentWindow, and save-set processing.

\subsection{Image Security}

It should be impossible for an untrusted client to retrieve the image
contents of a trusted window unless a trusted client takes action to
allow this.  We introduce the following defenses in support of this
requirement.

The restrictions on resource ID usage listed above prevent untrusted
clients from using GetImage directly on windows not belonging to
trusted clients.

If an untrusted client tries to set the background-pixmap attribute of
an untrusted window to None, the server will instead use a
server-dependent background which must be different than None.

The X protocol description of GetImage states that the returned
contents of regions of a window obscured by noninferior windows are
undefined if the window has no backing store.  Some implementations
return the contents of the obscuring windows in these regions.  When
an untrusted client uses GetImage, this behavior is forbidden; the
server must fill the obscured regions in the returned image with a
server-dependent pattern.

If an untrusted window has trusted inferiors, their contents are
vulnerable to theft via GetImage on the untrusted parent, as well as
being vulnerable to destruction via drawing with subwindow-mode
IncludeInferiors on the untrusted parent.  An untrusted window having
trusted inferiors can only occur at the request of a trusted client.
It is expected to be an unusual configuration.

\subsection{Property Security}

Unlike the other security provisions described in this document,
security for property access is not amenable to a fixed policy because
properties are used for inter-client communication in diverse ways and
may contain data of varying degrees of sensitivity.  Therefore, we
only list the possible restrictions the server may decide to impose on
use of properties on trusted windows by untrusted clients.  How the
server chooses which restrictions from this list to apply to a
particular property access is implementation dependent
\footnote{In the X Consortium server implementation, property access
is controlled by a configuration file; see the -sp option in the
Xserver(1) manual page.}.

The X Protocol property requests are ChangeProperty, GetProperty,
DeleteProperty, RotateProperties, and ListProperties.  For these
requests, the server can allow the request to execute normally (as if
it had been issued by a trusted client), ignore the request completely
(as if it were a NoOperation), or ignore the request except to send an
Atom error to the client.  Ignoring a ListProperties request means
replying that the window has no properties.  ListProperties may also
reply with a subset of the existing properties if the server is doing
property hiding; see below.  An ignored GetProperty request may reply
that the property does not exist, or that it exists but contains no
data.

The server may decide to hide certain properties on certain windows
from untrusted clients\footnote{The X Consortium server
implementation does not currently provide a way to hide properties.}.
If a property is to be hidden, it must be done consistently to avoid
confusing clients.  This means that for untrusted clients:

\begin{itemize}

\item That property should {\em not} be returned by ListProperties.
\item PropertyNotify events should {\em not} be sent for that property.
\item GetProperty on that property should reply that the property
does not exist (the return type is None, the format and bytes-after
are zero, and the value is empty).

\end{itemize}

For a property that the server is protecting but not hiding,
consistency must also be maintained:

\begin{itemize}

\item That property {\em should} be returned by ListProperties.
\item PropertyNotify events {\em should} be sent for that property.
\item GetProperty on that property should reply that the property
exists (if it really does) but the value is empty (return type and format
are their real values, and the "length of value" field in the reply
is zero).

\end{itemize}

\subsection{Miscellaneous Security}

If an untrusted client attempts to use ChangeHosts, ListHosts, or
SetAccessControl, the only effect is that the client receives an
Access error.

If an untrusted client attempts to use ConvertSelection on a selection
with a trusted selection owner window, the server generates a
SelectionNotify event to the requestor with property None.

\section{New Authorization Method}

This extension includes a new authorization method named
``XC-QUERY-SECURITY-1''.  Its purpose is to allow an external agent such
as the X firewall proxy to probe an X server to determine whether that
server meets certain security criteria without requiring the agent to
have its own authorization for that server.  The agent may use the
returned information to make a decision.  For example, the X firewall
proxy may choose not to forward client connections to servers that do
not meet the criteria.

To use this authorization method, the client (or proxy) sends
``XC-QUERY-SECURITY-1'' as the authorization-protocol-name in the
initial connection setup message.  The authorization-protocol-data may
be empty or may contain additional security criteria desribed below.
If the success field of the server's reply is Authenticate, the server
supports the security extension, and the server meets all specified
additional security criteria.  In this case, the client should resend
the initial connection setup message substituting the authorization
protocol name and data that should be used to authorize the
connection.  If the success field of the server's reply is anything
other than Authenticate, either the server does not support the
security extension, does not meet (or cannot determine if it meets)
all of the additional security criteria, or chooses for internal
reasons not to answer with Authenticate.  In this case, the client
should close the connection.

If the authorization-protocol-data sent with ``XC-QUERY-SECURITY-1'' is
not empty, it specifies additional security criteria for the server to
check, as follows.

\begin{arequest}{authorization-protocol-data}
\argdecl{policy-mask}{BITMASK}
\argdecl{policies}{LISTofSECURITYPOLICY}
\end{arequest}

The policy-mask field is any logical-OR combination of the constants
Extensions and SitePolicies.  For each bit set in policy-mask, there
is a SECURITYPOLICY element in policies.  The nth element in policies
corresponds to the nth 1-bit in policy-mask, counting upward from bit
0.

\begin{arequest}{SECURITYPOLICY}
\argdecl{policy-type}{\{Disallow, Permit\}}
\argdecl{names}{LISTofSTR}
\end{arequest}

For a SECURITYPOLICY corresponding to policy-mask Extensions, if
policy-type is Disallow the server is required to consider as insecure
all extensions given in names.  No policy is specified for extensions
not listed in names.  If policy-type is Permit the server may consider
only those extensions given in names to be secure; all other
extensions must be treated as insecure.  If these constraints are not
met, the server should not return Authenticate in the success field of
the reply.  Servers can but need not dynamically configure themselves
in response to an Extensions SECURITYPOLICY; a conforming server might
simply compare the policy with a compiled-in table of extensions and
their security status.


For a SECURITYPOLICY corresponding to policy-mask SitePolicies,
policy-type Disallow means the server must not have been configured
with any of the site policies given in names.  Policy-type Permit
means the server must have been configured with at least one of the
site policies given in names.  If these constraints are not met, the
server should not return Authenticate in the success field of the
reply.

SitePolicies provide a way to express new forms of security-relevant
information that could not be anticipated at the time of this writing.
For example, suppose the server is found to have a critical security
defect.  When a fix is developed, a site policy string could be
associated with the fix.  Servers with the fix would advertise that
site policy, and the X firewall proxy would specify that site policy
in a SECURITYPOLICY with policy-type Permit.

\section{Encoding}

Please refer to the X11 Protocol Encoding document as this section uses
syntactic conventions and data types established there.

The name of this extension is ``SECURITY''.

\subsection{Types}

AUTHID: CARD32

\subsection{Request Encoding}

% start marker
\begin{erequest}{SecurityQueryVersion}{0}{2}
\eargdecl{2}{CARD16}{client-major-version}
\eargdecl{2}{CARD16}{client-minor-version}
\areply
\eargdecl{1}{1}{Reply}
\eargdecl{1}{}{unused}
\eargdecl{2}{CARD16}{sequence number}
\eargdecl{4}{0}{reply length}
\eargdecl{2}{CARD16}{server-major-version}
\eargdecl{2}{CARD16}{server-minor-version}
\eargdecl{20}{}{unused}
\end{erequest}
% end marker

% start marker
\begin{erequest}{SecurityRevokeAuthorization}{2}{2}
\eargdecl{4}{AUTHID}{authorization-id}
\end{erequest}
% end marker

% start marker
\begin{erequest}{SecurityGenerateAuthorization}{1}{3 + (m+n+3)/4 + s}
\eargdecl{2}{CARD16}{m, number of bytes in authorization protocol name}
\eargdecl{2}{CARD16}{n, number of bytes in authorization data}
\eargdecl{m}{STRING8}{authorization protocol name}
\eargdecl{n}{STRING8}{authorization protocol data}
\eargdecl{p}{}{unused, p=pad(m+n)}
\eargdecl{4}{BITMASK}{value-mask (has s bits set to 1)}
\econstdecl{\#x00000001}{timeout}
\econstdecl{\#x00000002}{trust-level}
\econstdecl{\#x00000004}{group}
\econstdecl{\#x00000008}{event-mask}
\eargdecl{4s}{LISTofVALUE}{value-list}
\\
VALUES
\\
\eargdecl{4}{CARD32}{timeout}
\eargdecl{4}{}{trust-level}
\econstdecl{0}{SecurityClientTrusted}
\econstdecl{1}{SecurityClientUntrusted}
\eargdecl{4}{XID}{group}
\econstdecl{0}{None}
\eargdecl{4}{CARD32}{event-mask}
\econstdecl{\#x00000001}{SecurityAuthorizationRevoked}
\areply
\eargdecl{1}{1}{Reply}
\eargdecl{1}{}{unused}
\eargdecl{2}{CARD16}{sequence number}
\eargdecl{4}{(q+3)/4}{reply length}
\eargdecl{4}{AUTHID}{authorization-id}
\eargdecl{2}{CARD16}{data-length}
\eargdecl{18}{}{unused}
\eargdecl{q}{STRING8}{authorization-data-return}
\eargdecl{r}{}{unused, r=pad(q)}
\end{erequest}
% end marker

\subsection{Event Encoding}

% start marker
\begin{eerror}{SecurityAuthorizationRevoked}
\eargdecl{1}{0+extension event base}{code}
\eargdecl{1}{}{unused}
\eargdecl{2}{CARD16}{sequence number}
\eargdecl{4}{AUTHID}{authorization id}
\eargdecl{24}{}{unused}
\end{eerror}
% end marker

\subsection{Authorization Method Encoding}

For authorization-protocol-name ``XC-QUERY-SECURITY-1'', the
authorization-protocol-data is interpreted as follows:

% start marker
\begin{estruct}{authorization-protocol-data}
\eargdecl{1}{BITMASK}{policy-mask}
\econstdecl{\#x00000001}{Extensions}
\econstdecl{\#x00000002}{SitePolicies}
\eargdecl{m}{LISTofSECURITYPOLICY}{policies}
\end{estruct}
% end marker

% start marker
\begin{estruct}{SECURITYPOLICY}
\eargdecl{1}{}{policy-type}
\econstdecl{0}{Permit}
\econstdecl{1}{Disallow}
\eargdecl{1}{CARD8}{number of STRs in names}
\eargdecl{n}{LISTofSTR}{names}
\end{estruct}
% end marker

LISTofSTR has the same encoding as in the X protocol: each STR is a
single byte length, followed by that many characters, and there is no
padding or termination between STRs.

\section{C Language Binding}

The header for this extension is \verb|<X11/extensions/security.h>|.  All
identifier names provided by this header begin with XSecurity.

All functions that have return type \libtypename{Status} will return
nonzero for success and zero for failure.

% start marker
\begin{keeptogether}
\begin{cfunction}{Status}{XSecurityQueryExtension}
\cargdecl{Display *}{dpy},
\cargdecl{int *}{major\_version\_return},
\cargdecl{int *}{minor\_version\_return}
\end{cfunction}
% end marker

XSecurityQueryExtension sets major\_version\_return
and minor\_version\_return to the major and minor Security protocol
version supported by the server.  If the Security library is
compatible with the version returned by the server, it returns
nonzero.  If dpy does not support the Security extension, or if there
was an error during communication with the server, or if the server
and library protocol versions are incompatible, it returns zero.  No
other XSecurity functions may be called before this function.  If a
client violates this rule, the effects of all subsequent XSecurity
calls that it makes are undefined.

\end{keeptogether}

% start marker
\begin{keeptogether}
\begin{cfunction}{Xauth *}{XSecurityAllocXauth}
\cargdecl{void}{}
\end{cfunction}
% end marker

In order to provide for future evolution, Xauth structures are used to
pass and return authorization data, and the library provides ways to
allocate and deallocate them.

XSecurityAllocXauth must be used to allocate the
Xauth structure that is passed to XSecurityGenerateAuthorization.

For the purposes of the Security extension, the Xauth structure has
the following fields:

\begin{tabular}{lll}
\\ \hline
Type & Field name & Description \\ \hline
\typename{unsigned short} & \argname{name\_length} & number of bytes in name \\
\typename{char *} & \argname{name} & authorization protocol name \\
\typename{unsigned short} & \argname{data\_length} & number of bytes in data \\
\typename{char *} & \argname{data} & authorization protocol data \\ \hline
\\
\end{tabular}

The Xauth structure returned by this function is initialized as
follows: name\_length and data\_length are zero, and name and data are
NULL.

\end{keeptogether}

% start marker
\begin{keeptogether}
\begin{cfunction}{void}{XSecurityFreeXauth}
\cargdecl{Xauth *}{auth}
\end{cfunction}
% end marker

XSecurityFreeXauth must be used to free Xauth structures
allocated by XSecurityAllocXauth or returned by
XSecurityGenerateAuthorization.  It is the caller's responsibility to
fill in the name and data fields of Xauth structures allocated with
XSecurityAllocXauth, so this function will not attempt to free
them.  In contrast, all storage associated with Xauth structures
returned from XSecurityGenerateAuthorization will be freed by this
function, including the name and data fields.

\end{keeptogether}

% start marker
\begin{keeptogether}
\begin{cfunction}{Bool}{XSecurityRevokeAuthorization}
\cargdecl{Display *}{dpy},
\cargdecl{XSecurityAuthorization}{auth\_id}
\end{cfunction}
% end marker

XSecurityRevokeAuthorization deletes the authorization specified by
auth\_id, which must be a value returned in the auth\_id\_return
parameter of XSecurityGenerateAuthorization.  All clients that
connected with that authorization are be killed.  Subsequently,
clients that attempt to connect using that authorization will be
refused.

\end{keeptogether}

% start marker
\begin{keeptogether}
\begin{cfunction}{Xauth *}{XSecurityGenerateAuthorization}
\cargdecl{Display *}{dpy},
\cargdecl{Xauth *}{auth\_in},
\cargdecl{unsigned long}{valuemask},
\cargdecl{XSecurityAuthorizationAttributes *}{attributes},
\cargdecl{XSecurityAuthorization *}{auth\_id\_return}
\end{cfunction}
% end marker

XSecurityGenerateAuthorization creates a new authorization with the
specified attributes.  The auth\_in argument must be allocated by
XSecurityAllocXauth.  The name and name\_length fields of
auth\_in should be initialized to the authorization protocol name and
its length in characters respectively.  If there is authorization
data, the data and data\_length fields of auth\_in should be
initialized to the data and its length in characters respectivley.
The library does not assume that name and data are null-terminated
strings.  The auth\_in argument must be freed with
XSecurityFreeXauth.

The XSecurityAuthorizationAttributes structure has the following
fields:

\begin{tabular}{lll}
\\ \hline
Type & Field name & Mask \\ \hline
\typename{unsigned int} & \argname{trust\_level} & XSecurityTrustLevel \\
\typename{unsigned int} & \argname{timeout} & XSecurityTimeout \\
\typename{XID} & \argname{group} & XSecurityGroup \\
\typename{long} & \argname{event\_mask} & XSecurityEventMask \\ \hline
\\
\end{tabular}

These correspond to the trust-level, timeout, group,
and event-mask described in
the SecurityGenerateAuthorization protocol request.  The caller can
fill in values for any subset of these attributes.  The valuemask
argument must be the bitwise OR of the symbols listed in the Mask
column for all supplied attributes.
The event\_mask attribute can be None, XSecurityAuthorizationRevokedMask,
or XSecurityAllEventMasks.  In this revision of the protocol specification
XSecurityAllEventMasks is equivalent to XSecurityAuthorizationRevokedMask.
If the caller does not need to
specify any attributes, the attributes argument can be NULL, and the
valuemask argument must be zero.

If the function fails, NULL is returned and auth\_id\_return is filled
in with zero.  Otherwise, a pointer to an Xauth structure is returned.
The name and name\_length fields of the returned Xauth structure will
be copies of the name that was passed in, and the data and
data\_length fields will be set to the authorization data returned by
the server.  The caller should not assume that name and data are
null-terminated strings.  If no authorization data was returned by the
server, the data and data\_length fields will be set to NULL and zero
repectively.  The returned Xauth structure must be freed with
XSecurityFreeXauth; the caller should not use any other means free the
structure or any of its components.  The auth\_id\_return argument
will be filled in with the non-zero authorization id of the created
authorization.

\end{keeptogether}

% start marker
\begin{keeptogether}
The XSecurityAuthorizationRevokedEvent structure has the following fields:

\begin{tabular}{lll}
\\ \hline
Type & Field name & Description \\ \hline
\typename{int} & \argname{type} & event base + \\
		&		& XSecurityAuthorizationRevoked \\
\typename{unsigned long} & \argname{serial} & \# of last request processed by server \\
\typename{Bool} & \argname{send\_event} & true if this came from SendEvent \\
\typename{Display*} & \argname{display} & Display the event was read from \\
\typename{XSecurityAuthorization} & \argname{auth\_id} & revoked authorization id \\ \hline
\\
\end{tabular}

\end{keeptogether}
% end marker

\end{document}
